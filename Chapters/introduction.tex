\chapter{Introduction}
\section{Introduction}
Exoskeletons are a branch of wearable robotic devices that can be designed to meet unique requirements and provide varying levels of functionality. One branch of exoskeletons is rehabilitative exoskeletons for upper and lower bodies. Rehabilitative lower-limb exoskeletons enhance gait motion for people with reduced lower body control; this is an essential field of study for people with spinal cord injuries or other neurological injuries. Lower limb exoskeletons provide a method of enhancing and providing external support to the injured person. Increased exoskeleton research will improve the rehabilitation process for people with spinal cord injuries; however, this research is usually proprietary and expensive for new researchers.

The research presented in this dissertation aims to add additional controllers and models and provide open-source tools for exoskeleton development and research. This work is divided between the development of the controllers and open-source tools. The controller uses Learn from Demonstration (LfD) techniques and provides assistive torques to the person wearing the exoskeleton. It also uses human motion capture data to generate gait motions and control the exoskeleton joints. A simulation framework and controller for a lower-limb exoskeleton are presented. The aim and functionality of these exoskeletons are not targeted for assistive, recreational, or industrial use but instead intended for rehabilitative use. Therefore, the exoskeleton's controls will not be designed to handle arbitrary disturbances and collisions with the environment. Although the techniques presented here may be adapted for interaction with the environment, it is assumed that controllers are being used in a safe and open environment to avoid obstacles. This assumption is made because lower limb exoskeletons tend to be used in safe rehabilitation environments and not in daily living.

The contributions are separated into implementation and conceptual contributions. The implementation facilitated further testing and the development of the conceptual contributions described throughout the paper. The contributions are summarized below and will be discussed throughout the paper. 

\section{Primary Conceptual Contributions}
\begin{enumerate}[wide, nosep, labelindent = 0pt, topsep = 1ex]
     \item A trajectory controller based on human demonstration mocap data. A method using mocap data is presented to train a generalized trajectory for an exoskeleton to follow a learned trajectory; this allows for generalized gait motion that does not need to be retrained for each person. 
    \item Kinematic and dynamic models of a lower limb exoskeleton and a human wearer and their interactions. The exoskeleton will have sensors to measure the joint kinematics and the human torque to allow for cooperative controllers. The framework allows for the human effort commands to be sent to the system along with the typical robotic exoskeleton commands. 
    \item An optimal controller procedure using human demonstration for minimizing the control input. A method is presented using optimal control theory and human motion to develop a controller. This controller combines Learning from Demonstration (LfD) and optimization to build a controller from training data collection to torque generation. 
    \item Cooperative controller for shared control of a lower limb exoskeleton to handle the reduced torque of the person. The simulated torques applied to the model to represent the human wearer will be generated using a closed-loop controller that will be limited and time-varying in the amount of torque it can generate. The exoskeleton controller will account for the fatigue and reduction of torque. 
    \item Sensitivity analysis and tuning method for an Admittance Sliding Mode Controller (ASMC) to automatically tune the controllers' parameters. The sensitivity analysis shows the controller parameters' relative importance and their effect on the optimization cost function.  
\end{enumerate}

\section{Primary Implementation Contributions}
\begin{enumerate}[wide, nosep, labelindent = 0pt, topsep = 1ex]
    \item Open source toolkit for analyzing and parsing motion capture data with tools for gait analysis. This toolkit contains tools to handle marker data and rigid bodies.  
    \item Open source hardware of the lower limb exoskeleton for passive data collection. The exoskeleton will have sensors for detecting motion and be passive purely for data collection.
    \item A collection of joint data from a range of subjects, climbing stairs and walking, will be released open-source for community use.  
    \item  Implementation of an exoskeleton and human simulation for testing and implementation of the controller, assuming no ground collisions.
    \item A controller integrates the human and exoskeleton for a cooperative controller in simulation. The controller measures the human torque generated by a closed-loop controller while the exoskeleton provides the additional torque to follow a trajectory.  

\end{enumerate}