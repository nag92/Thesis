\chapter{Introduction}

Exoskeletons are a branch of human-machine robotic systems. Many different types of exoskeletons provide different functionality levels and are designed to meet different requirements. Rehabilitative lower-limb exoskeletons enhance gait motion for people with reduced lower body control; this is an essential field of study for spinal cord injuries or other neurological injuries. Lower limb exoskeletons provide a method of enhancing and providing external support to the person. Increasing the research in this field will improve the lives of people. Unfortunately, exoskeleton research is usually proprietary, expensive, and requires a vast knowledge of engineering and biomechanics. This research aims at solving several problems currently in the field and provide open source tools for exoskeleton development and research. This work is divided between the development of the controllers and open-source tools. The controller will learn from demonstration and provide assistive torques to the person wearing the exoskeleton.


This work focused on developing a simulation framework and a controller for a robotic lower-limb exoskeleton. The aim and functionality of these exoskeletons are  not targeted for assistive, recreational, or industrial use, but rather intended for rehabilitation use. Therefore, the exoskeleton's controls will not be designed to handle arbitrary disturbances and collisions with the environment. It is assumed that the controllers are being used in a safe and open environment to avoid obstacles, though the techniques presented here may be able to be adapted to enhance exoskeleton performance and interaction with the user in such environments. The controller will use human motion capture data to generate gait motions and control the exoskeleton joints. The contributions are separated into implementation and conceptual contributions. The implementations contributions enabled the testing and development of the conceptual development. The contributions are summarized below and will be discussed throughout the remainder of the paper. 

\section{Primary Conceptual Contributions}
\begin{enumerate}[wide, nosep, labelindent = 0pt, topsep = 1ex]
    \item  Modeling of an exoskeleton and human in simulation that couples the torque and integration. The exoskeleton will have sensors to measure the joint kinematics. The exoskeleton will measure the human torque to allow for cooperative controllers. The framework allows for the human effort commands to be sent to the system along with the typical robotic exoskeleton commands. 
    \item A trajectory controller based on human demonstration mocap data. A method using mocap data will be presented to train a generalized trajectory for an exoskeleton to follow a learned trajectory; this allows for generalized gait motion that does not have to be retrained for each person. 
    \item An optimal controller procedure using human demonstration for minimizing the control input. A method will be presented using optimal control theory and human motion to develop a controller. This controller combines learning from demonstration and optimal control for a controller optimized from the collection of training data to the torque generation. 
    \item Cooperative controller for shared control of a lower limb exoskeleton to handle the reduced torque of the person. The human torques will be generated using a closed-loop controller with that will be limited in the amount of torque it can generate. The exoskeleton controller will account for the fatigue and reduction of torque. 
\end{enumerate}

\section{Primary Implementation Contributions}
\begin{enumerate}[wide, nosep, labelindent = 0pt, topsep = 1ex]
    \item Open source toolkit for analyzing and parsing motion capture data with tools for gait analysis. This toolkit contains tools to handle marker data and rigid bodies.  
    \item Open source hardware of the lower limb exoskeleton for passive data collection. The exoskeleton will have sensors for sensing the motion and be passive purely for data collection.
    \item A collection of joint data from a range of subjects, climbing stairs and walking, will be released open-source for community use.  
    \item Implementation of an exoskeleton and human simulation for testing and implementation of the controller.
    \item A controller will be implemented that integrates the human and exoskeleton for a cooperative controller in simulation. The controller will measure the human torque generated by a closed-loop controller while the exoskeleton provides the additional torque to follow a trajectory.  
\end{enumerate}