\chapter{Exoskeleton Controllers}

\section{Introduction}

Controllers are a key aspect of lower limb exoskeletons; they allow the exoskeleton to move through desired motions. Different controllers have been developed to control the motion of the exoskeleton and human. A large aspect of these controllers is designed to consider the exoskeleton's dynamics model, allowing the non-linear dynamics to be canceled out when computing the control input. LARRE is assumed to be non-linear and modeled by \autoref{eq:Dyn} where $M$ is the mass-inertia matrix, $C$ is the Coriolis terms, $G$ is the gravity terms. This is similar to other models discussed in \cite{huo2016active} \cite{vantilt2019model}.

\begin{equation}
    \tau = M(q) \ddot{q} + C(q, \dot{q}) \dot{q} + G(q)
    \label{eq:Dyn}
\end{equation}


Several assumptions are made in the design of the controllers. First, the ground force reactions cannot be measured, which prevents feedback from the step. Second, human effort can be measured, and this is directly done through the simulation package. This measurement can be replicated on a physical exoskeleton using force sensing resistors or IMUs in the legs of the system. Third, the exoskeleton gait speed is slow. Gait training using an exoskeleton has been shown to improve the gait speed of paraplegics. The average walking speed after training was on average $0.43m/s$ \cite{khan2019retraining} compared to the average walking speed of a person is $1.5m/s$ \cite{fitzpatrick2006another}. A slow gait can be defined between $0.8-1.0m/s$. The expected gait speed is about half of the low end of a typical slow gait \cite{walsh2007quasi}; this means that the expected gait speed is on average less than $1/3$ of a normal gait speed. 

Two different controllers are presented; the first is an iLQR controller, the second is a cooperative controller. The iLQR controller takes advantage of learning from demonstration and optimal controller theory to generate an optimal control signal to control the exoskeleton. The cooperative controller is designed to provide assistive control commands to combine the person's torques and the exoskeleton. It enables the exoskeleton only to provide the difference in torque between the required and what the person provides. 

All the controllers are implemented using the control architecture discussed in \autoref{sec:controlarchitecture}—the modular control system allowed for controllers to be quickly implemented and tested. The RBDL library allowed for the fast computation of the dynamics of the exoskeleton allowing for real-time control. 

% The controller presented in \cite{martinez2017controller} was implemented on the Indigo exoskeleton and used a combination of force fields and human demonstration data to control the movement of the exoskeleton joints. This method did not consider the mass of dynamics of the exoskeleton and was designed to work on stroke patients with unilateral control. 


\section{iterative Linear Quadratic Regulator}


\subsection{Overview of iLQR Controller}

The objective of this controller was to close the loop between the reproduction models and the controller. Current learning from demonstration methods has only been combined with linear quadratic regulators; this limits the applicability of processes since most robotic systems have non-linear dynamics. An iterative linear quadratic regulator is used to find an optimal control signal to drive the exoskeleton joints through the desired trajectories. A PD controller is added as a feed-forward control component for unmodeled dynamics and optimized using the Bayesian Information Criterion. We show how the trajectory is learned, and the control signal is optimized by reducing the required bins for learning. The framework presented produces optimal control signals to allow the exoskeleton's legs to follow human motion demonstrations.

The Linear Quadratic Regulator (LQR) is a well-known method that provides optimally controlled feedback gains to enable the closed-loop and high-performance control \cite{kirk2004optimal}. The limitation of LQR is that it can only be implemented on a system with linear dynamics and has a linear cost function. In \cite{TPGMM_calinon2016}, Calinon \textit{et. al} used TPGMM/GMR with an LQR controller to develop a minimal intervention controller. This controller was used to control a Kuka robotic arm \cite{schreiber2010fast} through a series of movements. This method leads the inspiration to use a non-linear controller instead of the linear one used previously.  

The Iterative Linear Quadratic Regulator (iLQR) is a non-linear version of the LQR controller. The iLQR is an iterative process that uses a Taylor approximation of the dynamics and cost function to find a local linear model. The dynamics and cost function are linearized in the forward pass of the system using a shooting method, while like the vanilla LQR, the backward pass calculated the optimal gain and cost \cite{iLQR_paper}. Differential Dynamic Programming(DDP) is a similar method to iLQR; in classical DDP, the second-order terms are costly operations \cite{iLQR_tassa2014}  \cite{iLQR_Zachary2016}.

This modification to LQR allows the control of non-linear system control problems; this is useful because it expands the systems the LQR can be applied to, including biological movement system \cite{iLQR_Li2004} and online trajectory optimization \cite{iLQR_tassa2012}. iLQR compared to ODE solvers, gradient descent methods, and differential dynamic programming converges substantially faster and finds slightly better solutions \cite{iLQR_Li2004}. 

iLQR controllers have been implemented on a wide variety of systems. In \cite{car} they used a modified form of iLQR called constrained iLQR to control the motion of a car. The car was subjected to several state and control constraints. iLQR also allows for the control of humanoid robots. Tassa \textit{et. al} used iLQR to control an HRP-2 robot's motion by controlling the joint angles \cite{iLQR_tassa2014}.

\subsection{Methods and Implementation}

The proposed approach is split into several phases; demonstration, encoding, and optimization. During the demonstration phase, gait data is collected, and the gait cycles are parsed to extract the joint angles. The demonstrations are encoded using TPGMM/GMR as discussed in \autoref{sec:lfd}. In the demonstration and encoding phases, the trajectories are learned and encoded. The controller is tested on the LARRE model in AMBF as discussed in \autoref{chap:sim}. 

During the optimization phase, the control inputs calculated drive the system along the trajectory. There are two steps in the iLQR algorithm; a forward pass and a backward pass. The simulated forward LARRE is simulated forward along the trajectory using a dynamic model in the forward pass. 

Runge Kutta 4 (RK4) integrates the system forward to obtain the next state of the system \cite{dit2017runge}. RK4 is a numerical integration method that perturbs the system about a point and uses an average response average value.  In the backward pass, the system is solved backward to update the control parameters. A modified version of the open-source library\footnote{https://github.com/anassinator/ilqr} was used and implemented.  \autoref{fig:ilqrDiagram} shows a diagram of how the iLQR algorithm works. The algorithm iteratively continues back and forth until the cost $J$ converges, indicating that the control signal $u$ can drive the system along the desired trajectory. 

\begin{figure}[h]
    \centering
    \includegraphics{images/controllers/ilqr2.png}
    \caption[iLQR Learning Loop Diagram]{Diagram of how the iLQR algorithm works with the forward pass and backward pass. The loop exists when the difference in current cost and previous cost is below a tolerance. }
    \label{fig:ilqrDiagram}
\end{figure}



\autoref{eq:system_dyn} defines the  generalized non-linear system dynamics. The cost function is the sum of the running cost and the terminal cost shown in \autoref{eq:cost}. This paper uses a linear cost function to take advantage of the TPGMM/GMR process shown in \autoref{eq:mycost}, where $\tilde{x}_k = x_k - x^{d}_k$. This paper's cost function is designed to follow a reference trajectory $x^{d}$ using GMR. The $Q_k$ varies along the path and  are calculated by the TPGMM algorithm ($Q_k$ =  $\Sigma_k^{-1}$). The  $Q$ matrix is the weight for transitioning, and the $R$ matrix is the weight of the control. 

\begin{equation}
     x_{k+1} = f(x_k,u_k) 
     \label{eq:system_dyn}
\end{equation}

\begin{equation}
    J(x,U) = \ell_f (x_N) + \sum_{k=0}^{N-1} \ell(x_k, u_k) 
    \label{eq:cost}
\end{equation}

where,
\begin{equation}
    \begin{split}
            \ell(x_k, u_k) &= \tilde{x}_k^T Q_k \tilde{x}_k + u_k^T R u_k \\
    \ell_f(x_N) &= x^{T}_{N} Q_N x_{N}
    \end{split}
      \label{eq:mycost}
\end{equation}

The value function is found using \autoref{eq:value} which is minimized over the entire control sequence. Using calculus of variations, \autoref{eq:deltaQ} is found which is decomposed into  \autoref{eq:deltaQDecomp}, where $ A =\partial f / \partial x$ and  $B = \partial f / \partial u$. The particles derivatives are calculated in the forward pass at each time step using a finite difference method \cite{iLQR_Zachary2016}. 


\begin{equation}
    \begin{split}
        Q(x,u) &= \ell (x,u) + V(f(x,u,i+1) \\
        V(x,i) &= \min\limits_{u} Q(x,u)
    \end{split}
    \label{eq:value}
\end{equation}


\begin{equation}
     \delta Q = 
     \frac{1}{2}
     \begin{bmatrix}
     1 \\
     \delta x \\
     \delta u
     \end{bmatrix}^T
       \begin{bmatrix}
        0       & Q^T_{x} & Q^T_{u}  \\
        Q_{x}   & Q_{xx} & Q_{xu}  \\
        Q_{u}   & Q_{ux} & Q_{uu} 
    \end{bmatrix}
    \begin{bmatrix}
     1 \\
     \delta x \\
     \delta u
     \end{bmatrix}
        \label{eq:deltaQ}
\end{equation}

\begin{equation}
    \begin{split}
        Q_x &= \ell_x + A^T V'_x \\
        Q_u &= \ell_u + B^T V'_x \\
        Q_{xx} &= \ell_{xx} + A^T V'_{xx}A \\
        Q_{uu} &= \ell_{uu} + B^T V'_{xx}B \\
        Q_{ux} &= \ell_{ux} + B^T V'_{xx}A \\
    \end{split}
    \label{eq:deltaQDecomp}
\end{equation}



The optimization finds the total cost and the optimal control gains for the system. The control sequence is found using \autoref{eq:control}, where $K=-Q_{uu}^{-1}Q_{ux}$ and $k=-Q^{-1}_{uu} Q_{ux}$. The $\alpha$ term is a linear search term to ensure convergence of the system, and $\hat{u}_k$ is the nominal control input. 


\begin{equation}
    u_k = K_k (x_k - \hat{x}_k) + \alpha k_k + \hat{u}_k
    \label{eq:control}
\end{equation}

One of the primary problems with iLQR is that it does not allow for real-time control. The control command is calculated offline and applied online due to the computational time for the forward and backward loops. The calculation can take up to several seconds or minutes depending on the degrees of freedom, length of trajectory, and computational power of the computer used to train. To circumvent this problem, a PD controller was added into the loop to allow for error tracking in real-time; the PD controller handles errors in un-modeled dynamics suck as joint friction and damping \cite{iLQR_tassa2014}. The iLQR torque acted as a feed-forward term driving the system, and the PD controller handled path deviation errors. \autoref{fig:controller} show the control diagram. 

\begin{figure}[H]
    \centering
    \includegraphics[width=\linewidth]{images/controllers/iLQR.png}
    \caption[iLQR Control Diagram]{Control diagram of the exoskeleton. The trajectory is found using TPGMM. The iLQR provides a feedforward control input. The PD controller removes un-modeled dynamics of the system }
    \label{fig:controller}
\end{figure}

As discussed above, the TPGMM process finds optimal values for the $Q_i$ along the trajectory, which is the weighting of the system's state. However, this does not provide insight into the form of the $R$ matrix, which is the weighting of the control input into the system. These values are critical because they determine the amount of effort applied at every point along the trajectory. The shape of the $R$ matrix for this application  $6 \times 6$ diagonal matrix.


The first three diagonal elements of the $R$ matrix are correlated to the control input of the left leg (\textit{hip}, \textit{knee}, \textit{ankle}). The other three diagonal elements are related to the control input of the right leg (\textit{hip}, \textit{knee}, \textit{ankle}). This paper assumes symmetry of the joints for the left and right legs i.e. $\textit{hip}_R$ == $\textit{hip}_L$.  This assumption is possible because each leg would have similar masses and controlled with identical motors. In addition, the minor differences can be supplemented by the $Q$ matrix during the iLQR training.  

To find the $R$ matrix's values; the values were iteratively changed to find a matrix that minimizes cost $J$ defined in \autoref{eq:R_cost}, where $N$ is the number of points for output, $x_i$ are the points on the desired trajectory, and $\hat{x_i}$ are the points on the actual trajectory. 
 \cite{chai2014root}.

\begin{equation}
    J = \sqrt{\frac{\sum_{i=1}^N(x_i-\hat{x_i})^2}{N}}
    \label{eq:R_cost}
\end{equation}

The high dimensionality and non-linear dynamics make it challenging to weight values of the matrix \cite{park2012multi}. The complexity of the motor behaviors also increases the difficulty of finding the $R$ matrix. Therefore, a brute-force method was implemented to go through values in different magnitudes and select the optimal result. The control signal was tested by forward integrating using RK4.  

% \autoref{fig:error_digram} show the optimal trajectory and one unfit case:

% \begin{figure}[H]
%     \centering
%     \includegraphics[width=\linewidth]{Images/effor_2.png}
%     \caption{Joint Angles for the trajectories. The blue line is the desired motion, the red line is well fit trajectory, and the green line is a poorly fit trajectory.}
%     \label{fig:error_digram}
% \end{figure}

% Changing the values of the $R$ matrix significantly affects the replication of the trajectory. The $R$ matrix has to be tuned in order to replicate the desired trajectory.  

\subsection{Results and Discussion}



The path and $Qs$ are generated by the TPGMM/GMR process and initialize the iLQR controller algorithm. As the name implies, the iLQR algorithm is an iterative possesses that breaks when the cost $J$ converges.  \autoref{fig:cost} shows the converges of the cost for each iteration. The cost coverage's from $\sim47.5 \rightarrow \sim27.5$ after 6 iterations. The convergence of the iLQR indicates the error between the desired and actual have been reduced, and the efforts along the trajectory have been minimized. 

 

\begin{figure}[h!]
    \centering
    \includegraphics[scale=0.22]{images/controllers/Cost_plt3.png}
    \caption[iLQR controller Convergence]{Converges of the iLQR controller cost.}
    \label{fig:cost}
\end{figure}


\autoref{fig:comparison} shows a comparison of the exoskeleton joints to the reference trajectory. The orange line references the trajectory, and the blue line is the path the exoskeleton joints traveled. LARRE's joints were able to track the desired motion. 


\begin{figure}[h!]
    \centering
    \includegraphics[scale=0.27]{images/controllers/compare_traj.png}
    \caption[iLQR controller trajectory]{Comparison of the actual joint angles to the reference trajectory.}
    \label{fig:comparison}
\end{figure}


\autoref{fig:comparisonTorque} shows a comparison of the joint effort over the trajectory. Both the iLQR feed forward term and the total torque (iLQR+PD) are presented. Additionally, the effort of a pure PD controller is presented for comparison of effort. This is not the same PD effort used for the total torque ( \textit{orange} +  \textit{green} $\neq$  \textit{blue} ). The iLQR term can closely follow the total torque indicating that most of the control command is provided from the iLQR controller, not the PD term. Additionally, the PD controller control input is noisy and has greater efforts than the iLQR control signal showing that the iLQR can produce lower torque than the vanilla PD controller. 

\begin{figure}[h!]
    \centering
    \includegraphics[scale=0.35]{images/controllers/torque_compare.png}
    \caption[Torque Comparison of the iLQR controller]{Comparison of the PD controller to the iLQR torque and the total torque}
    \label{fig:comparisonTorque}
\end{figure}

\section{Cooperative Controllers}


\subsection{Overview Cooperative Controllers}

Assistive controllers have long been implemented in rehabilitative exoskeletons. This includes the MIT-MANUS \cite{ju2005rehabilitation}, BLEEX\cite{kazerooni2006hybrid}, HAL\cite{kawamoto2004power} and many others robotic systems including several exoskeletons \cite{kim2012admittance,ott2010unified,huo2011control}. These systems use assistive controllers to help guide the person's arm or leg through some desired trajectory or complete tasks. This method is typically used because the person has difficulty generating the necessary torques due to medical problems such as a stroke or spinal cord injury. The difficulty in designing these systems is in handling the non-linearity and distances from being connected to a person, which the controller will have to overcome. Many of these controllers implement either impedance or admittance models. Admittance controllers transform forces and torques to the desired position and orientation. Impedance controllers are the mirror calculating displacements from applied forces and torques. They allow for virtual stiffness, which is useful for controlling the human-robot interactions \cite{keemink2018admittance}. 

The admittance controller has shown better results when implemented into an exoskeleton. They are easier to implement and can measure the intention of the user. Admittance controllers allow for the adjustment of the stiffness to meet the on-the-fly demand by either being more or less aggressive in the supplied force \cite{aguirre2007active,newman1994stable}. This problem is solved by using variable admittance, where the parameters are adjusted based on the user's intention and ability.

Oh \textit{et. al} proposed a generalized framework for assistive controllers \cite{oh2015generalized}. The proposed controller combined a model-based method with feed-forward disturbance rejection—this method models the interaction as a disturbance to the system instead of an interactive force.  Additionally, a PD controller was used, in contrast to a Sliding Mode Controller (SMC), which has been shown to have better performance in controlling non-linear systems. They both have robust and adaptive properties that are ideal for assistive systems \cite{slotine1991applied}.

SMC is a popular method in assistive controllers due to their insensitivity to disturbances and handling of non-linearity better than traditional PD controller \cite{nasir2010performance} \cite{sanngoen2020review} \cite{fischer-SMC}. SMC works by using a switching function to drive the system along a sliding surface. Torabi \textit{et al.} used a model-based SMC controller to drive a lower limb exoskeleton. This controller also used an adaptive admittance controller along with the SMC to adjust to the user intention ability \cite{torabi2018robust}. A similar method of model-based SMC was used by Babaiasl \textit{et. al} to control an upper limb exoskeleton \cite{babaiasl2015sliding}. In \cite{long2016robust} a genetic algorithm was used to tune the parameters of an SMC for a lower limb exoskeleton. The parameters were tuned in MATLAB/SIMULINK. While successful, they did not model a connection between the human and exoskeleton and did not use an adaptive admittance controller. 

\begin{figure}
    \centering
    \includegraphics{images/controllers/SMC.png}
    \caption{Sliding mode control and the switching phase}
    \label{fig:SMC}
\end{figure}

\subsection{Development of Cooperative Controller and Tuning Methods}

In this section a method of developing a closed loop assastive controller will be discussed. First a simplified system is presented so that the controller can be developed. The simplified system constitutes of two double pendulums connected by a mass-spring dampener at each link. Then a method of tuning the parameters is presented along with the effects on effort reduction and system misalignment.

\autoref{fig:double_pend} illustrates the simplified system. Because one set of pendulums is being assisted by the other set, one pendulum is the \textit{assistie} and the other pendulum is the \textit{assisitor} system. The \textit{assisitor} system provides additional torques to help the \textit{assistie} to move through some desired motion. This model and controller are based on the work by Tu \textit{et. al} \cite{tu2020adaptive}, however the system was modified so that \textit{assistie} system is driven by a closed-loop controller. These changes add complexity to the system since the \textit{assistor} system has to handle the errors that arise in the attached closed-loop controller. With two closed-loop systems connected, the controller commands are magnified when handling the errors \cite{tu2020adaptive}. \autoref{fig:controlDiagram} shows the control diagram. The first model has a pendulum that is controlled by a simple closed-loop PD controller. The second model is controlled by an Admittance-Sliding Mode Controller (A-SMC), which has useful properties for handling non-linear dynamics and disturbances. Additionally, a gradient descent method is used to auto-tune the PD controller parameters and the A-SMC. 

The problem with A-SMC controllers is the large number of parameters that need to be tuned. These parameters can have non-linear and hard-to-determine effects on the response of the system \cite{slotine1983tracking}. These parameters include SMC parameters and the variable admittance model parameters. Both of which scale with the dimensions of the system. Having a comprehensive method to determine these parameters allows the system to be tuned automatically.  

\begin{figure}
    \centering
    \includegraphics[scale=1.5]{images/controllers/double_pend.png}
    \caption{Double connected pendulum. Spring-dampeners connect the two pendulum (\textit{assistor} and \textit{assistie}) }
    \label{fig:double_pend}
\end{figure}

\begin{figure*}
    \centering
    \includegraphics[scale=0.9]{images/controllers/SMC_control_diagram_overview.png}
    \caption{Control diagram for A-SMC }
    \label{fig:controlDiagram}
\end{figure*}

\autoref{eq:CooPdyn} describes the dynamics for the \textit{assistor} (abbreviated has \textit{tor}) and \textit{assistie} (abbreviated has \textit{tie}) respectively.  It should be noted that $q$ is the joint state for the respective model that they describe. The $F$ terms are the forces generated as a result of being coupled by a spring dampener described by \autoref{eq:coupling}. Additionally, $J$ is the Jacobian of each link to the connection point of the spring dampener system. Although any point on the link could be used to calculate the Jacobian matrix, the connection point is at the end of each link. This assumption does change the process, only the calculation of the Jacobian matrix. It can later be adjusted to the location of the straps of an exoskeletons.

\begin{equation} 
\begin{aligned}
    M_{tor}(q) \ddot{q} + C_{tor} (q,\dot{q}) + G_{tor}(q) &= \tau_{tor} + J_{tor}^T F \\
    M_{tie}(q) \ddot{q} + C_{tie} (q,\dot{q}) + G_{tie}(q) &= \tau_{tie} + J_{tie}^T F
\end{aligned}
    \label{eq:CooPdyn}
\end{equation}

\begin{equation}
    F = K ( \vec{x}_{tor} - \vec{x}_{tie} ) + B (\dot{ \vec{x}}_{tor} - \dot{ \vec{x}}_{tie} ) 
    \label{eq:coupling}
\end{equation}


The \textit{assisitie} system was controlled by \autoref{eq:PDcnrl}. This is simple PD control framework. $K_p$ and $K_d$ are gain matrixs and $\vec{q}_d$ is the desired position. This torque however was capped to not exceed an absolute max toque value. This was accomplished using the following method, $| \vec{\tau}_{tie}|> \vec{\tau}_{max} \rightarrow \vec{\tau}_{tie} = sign(\vec{\tau}_{tie})*\vec{\tau}_{max}$.  This did not allow the controller to generate the required torque. 

% \begin{equation}
%         \tau_{tie} = K_p( \vec{q}_d - \vec{q}_{tie} ) + K_d ( \dot{\vec{q}}_d - \dot{\vec{q}}_{tie} ) 
%     \label{eq:PDcnrl}
% \end{equation}

The admittance controls the virtual dynamics of the system and how the systems interact \cite{faulring2005haptic}. If the \textit{assastie} system is capable of following the desired motion, the \textit{assasitor} system will be less aggressive. This interaction is controlled by \autoref{eq:addmittance}, where $x_a$ is the location of the virtual system and $x_d$ is the location of the point on the pendulum link. A separate spring-dampener system is on each of the links of the system. The virtual system is defined by the following terms $M_d$, $B_d$, and $K_d$ are inertia, dampening, and stiffness, respectively. As previously stated, it is desirable to have these parameters adjust on the fly to the intention and ability of the \textit{assistie} system. The variability is controlled by \autoref{eq:intention} and \autoref{eq:varible}. Here $\alpha_{n,p}$ and $\gamma_{n,p}$ are tuning variables for the damping and stiffness of the admittance controller. The admittance will control how quickly and aggressively the \textit{assistor} system will respond. If the stiffness or dampening is too large, the model will not track the desired motion.   

\begin{equation}
    \begin{aligned}
        \tau_{int} &= M_d \ddot{e}_a + K_d e_a + B_d \dot{e}_a  \\
        e_a &= x_a - x_d 
    \end{aligned}
    \label{eq:addmittance}
\end{equation}

\begin{equation}
    \begin{aligned}
         intent &= bool ( sign(T_h) == sign(\dot{q}_d) ) \\
         \mu &= \Big|\frac{T_h}{T_{id}} \Big|
    \end{aligned}
    \label{eq:intention}
\end{equation}



\begin{equation}
    \begin{bmatrix} K_d \\ B_d \end{bmatrix} = \begin{cases}
        \begin{bmatrix} K_{p} \\ B_{p} \end{bmatrix} + \mu  \begin{bmatrix} \gamma_p  \\ \alpha_p \end{bmatrix}, intent = 1 \\
        \begin{bmatrix} K_{n} \\ B_{n} \end{bmatrix} - \mu  \begin{bmatrix} \gamma_n  \\ \alpha_n \end{bmatrix}, intent = 0
  \end{cases}
  \label{eq:varible}
\end{equation}


The A-SMC controller suffers from an abundance of tunable parameters. The admittance controller and SMC have parameters requiring synergistic tuning since these systems are connected.  A gradient descent method is used to find optimal parameters based on various cost functions and constraints for the controller.  Gradient descent is used since it should arrive quickly to a solution  \cite{piltan2012performance} \cite{wang1996course}. The goal is to minimize the objective function by updating parameters. This method will iterate until the objective function converges to a locally optimal location. The tunable parameters for this controller are the variable admittance parameters ($K_{n,p}$, $B_{n,p}$, $\alpha_{n,p}$, $\gamma_{n,p}$) and the sliding mode controller variables ($\rho$, $\Lambda$, $\beta$).  Another important variable is the limit of the torque that the \textit{assistor} system can provide, and this limitation is important since physical real systems cannot produce infinite torque. Motors and gearboxes present real limitations on the system. This method allows for the maximum torque to be used, has a hard limitation, and allows for the minimal limit to be found.


The optimization was calculated using Simulink Response Optimization\footnote{https://www.mathworks.com/help/sldo/response-optimization.html}. Several cost functions were tested to find the optimal performance, and the model was trained to follow a polynomial trajectory. Root mean squared error (RMSE) was used for the cost function shown in \autoref{eq:RMSE}, where $x$ is the observation and $\hat{x}$ is the reference, either position or velocity. The error is summed along both the trajectory and the joints to get a scalar value. If both the position and velocity RMSE is used, it is then summed.

The gain parameters $K_p$ and $K_d$ in \autoref{eq:PDcnrl} were also tuned using the same method. The controller assumed that no assistance was provided by the \textit{assistor} system. These gains were set for the \textit{assistie} controller in the closed-loop system.

\begin{equation}
    e = \sqrt{ \frac{ \sum_1^N ( \hat{x}_i - x_i )^2 }{N}  }
    \label{eq:RMSE}
\end{equation}


\section{Contributions}

Several contributions were made to the field of controller development. Both the iLQR controller and the A-SMC cooperative controller introduced new methods that expand on previous applications and research. Both the controller have application outside the field outside of lower limb exoskeletons. The generalized forms of the controller can be used for assastive control applications and to generate optimal control signals for non-linear robotic systems. 


The iLQR controller introduced a new method of integrating an optimal controller with learning from demonstration to control a lower limb exoskeleton. This controller method allows a trajectory to be learned from multiple demonstrations and uses a non-linear dynamic model to learn an optimal control signal.  Using a non-linear model allows for the complex dynamics to be encoded into the control signal. The PD controller allows feedback to account for non-modeled dynamics. 



The new tuning method of an A-SMC controller was presented, allowing for the parameters to be automatically determined. This method was shown to improve the state space of the SMC. Using a position and velocity cost function allows for the gains to be automatically tuned to the system. This generalization of the tuning method allows it to be used for more complex systems with more joints or complex dynamics.  Additionally, it was shown that the controller was able to handle varying alignment and involvements of the \textit{assitie} system; this is important since these controllers are often incorporated into \textit{assistive} exoskeleton systems. 

