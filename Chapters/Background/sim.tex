\section{Exoskeleton Simulations}
Physics-based and graphical simulation is a crucial aspect of exoskeleton development. It allows the development of multiple control algorithms for the exoskeleton without the difficulties of experimentally testing the exoskeleton on a person. Simulation allows the system to be modeled, controlled, and tested in a virtual environment before being implemented in a physical experimental environment. It is a complicated method and essential for controller development \cite{ZLAJPAH2008879}.   

Sit-to-stand motion is one of the many movements modeled in simulation to model dynamics and test control algorithms. A FES based muscle model can be used to calculate the stimulation patterns to drive the leg muscles as described in \autoref{sec:FES}.
A tracking algorithm to model the human's upper body was designed in Simulink/Matlab. The PID controller is implemented in Matlab/Simulink \footnote{https://www.mathworks.com/products/matlab.html} controlled the orientation of the human's torso. 

Yan \textit{et. al} combined the Automated Dynamic Analysis of Mechanical Systems \footnote{https://www.mscsoftware.com/product/adams} (ADAMS) and Matlab/Simulink to compare the trajectory of an exoskeleton controlled with a PID controller to their developed sliding mode control algorithm \cite{Yan_2017}. ADAMS is the simulation environment for the physical model, while MATLAB calculates the algorithm for the controllers. ADAMS was also used by Copilusi \textit{et. al} to perform dynamic analysis of a light-weight lower-extremity exoskeleton, where simulation results presented ample performance for walking rehabilitation  \cite{copilusi2014}. ADAMS also performed the numerical simulation \cite{geonea2017design} of the lower-extremity exoskeleton and compared the experimental walking of healthy subjects. 

While all these approaches have different objectives, they all successfully use simulation to model and develop controllers for their systems; this shows the versatility and capability of using simulation for exoskeleton development. 
