\subsection{Upper Limb Exoskeletons}
While not the paper's focus,  the development of upper limb rehabilitative exoskeletons should be noted, there have been several robotics systems developed for upper limb rehabilitation. These systems are focused on repetitive motion with assistive forces \cite{rehmat2018upper} \cite{krebs2013rehabilitation}, this is applicable to lower limb exoskeletons as well since they need to provide assistive force to help maintain a gait motion. 

One of the early robots developed was the \textbf{MIT-Manus} \cite{krebs2004rehabilitation}. This robot has two DoF that can move in the horizontal plane, which eliminates gravity. There is a visual feedback system that helps the patient engage with the rehabilitation. The robot also provides assistive torque to help the person move through motions.    

\begin{figure}
    \centering
    \includegraphics[scale=0.5]{images/background/MIT-MANUS.png}
    \caption[MIT-MANUS]{MIT-MANUS \cite{MIT-Manus}}
    \label{fig:my_label}
\end{figure}


Additionally, the \textbf{UL-EXO7} is a seven DoF robotic arm that provides assistive forces and a full range of motion. This robot, like the MIT-Manus, also is used to play an interactive game. The initial trial has been shown to have functional patient outcomes. 

\begin{figure}
    \centering
    \includegraphics[scale=0.5]{images/background/ULEXO7.png}
    \caption[UL-EXO7]{UL-EXO7 \cite{byl2013chronic}}
    \label{fig:ULEXO7}
\end{figure}


Another rehabilitation arm exoskeleton is the \textbf{T-WREX}. Unlike some of the other systems, the T-WREX is pneumatically powered \cite{TWREX}; this had the advantage of large non-linear forces with low onboard mass. This system has also been shown to have successful rehabilitative outcomes \cite{housman2007arm}.
