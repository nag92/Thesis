\section{Future Work}

The future work for this research includes completing the physical exoskeleton system. The Maxon motors need to be integrated with the harmonic gearbox and LARRE's hips and knees, allowing the exoskeleton to be used actively. The design for the motor plates is complete but needs to be manufactured. The motor coupler between the Harmonic gearbox and the Maxon motor has been designed. Additionally, an extra un-actuated degree of freedom should be added to the hip to allow some side-to-side motion, which would help directional change and turning. 

The Xilinx control board also needs to be completed and programmed; while this board is built, the motor drivers need to be integrated, and low-level code needs to be developed to control the motors. Additional development needs to be done to use the sensors to measure the state of LARRE. Once the motors and controllers have been integrated into LARRE, a study on typically abled people should be conducted. The cooperative controllers need to be tested and compared to the simulation models; this will also require writing the code to run on real-time OS on an embedded system. All human studies should be conducted with extreme caution to avoid injury. 

The bio-knee is a promising area of study. The knee can be customized for individuals by studying MRI scans and motion capture trails of healthy patients. The MRI can show the difference in alignment between a pin joint knee and the customized bio-knee model. Additional research needs to be done to examine the relationship of the femur and tibia head shapes to height and gender so that expensive MRIs do not necessarily have to be conducted to get the correct scaling parameter. Additionally, the limitation of mocap needs to be overcome to measure the rotation to translation relationship effectively. However, this will be complex to accomplish due to the inherent complications of skin-based markers and the difficulty of calculating the instantaneous center of rotation with the frame to frame marker locations.  

% While mocap is not well suited for calculating the rotation-translation motion of the knee; it could be used to calibrate the joint placement of the exoskeleton, which is vital since misalignment of the exoskeleton joint could cause injuries to the person. In \cite{zanotto2015knee}, the effects of joint misalignment that lower extremity exoskeletons have on the knee joint are examined. It was found that interaction forces and torques were introduced due to the joint misalignment in the robot-human interactions. It was also noted that in the design of a lower extremity exoskeleton, it is preferable to have a longer robot thigh link over a shorter robot thigh link, placing the human knee above the joint of the exoskeleton. It can also prevent the exoskeleton from effectively transferring and controlling the torque effectively while worn by a patient \cite{cempini2012self}.

Additionally, the knee should be dynamically tested using a motion capture system and force plates. The dynamics and kinematics of the knee can be measured. An IRB has to be written for this study. Strap connectors have to be designed and built to attach to the person. The IRB trial should include approximately ten subjects walking across the room. The gait cycles can be compared by comparing the person walking with and without the bio-knee. 

The simulation framework can be included to allow ground-level walking. The AMBF simulation framework is still new and under constant development, with new bugs being discovered and updates being pushed constantly. The new updates may be used to fix the frictional problem between the exoskeleton's feet and the ground plane. Once this is complete, a more complex controller can be tested in the simulation environment. The controllers should have the ability to handle the ground force reactions since they are robust to disturbance. 

The cooperative controllers and LfD methods presented in this dissertation have broader applications out of the field of lower limb exoskeletons, including robotic surgical systems and industrial robotic systems. A two Dof pendulum should be developed to simulate and test the controller on a system that closely resembles a leg. This system can be improved over the presented system by using better motors with more accurate encoders, allowing more refined control over the robotic system. 

The physical system to test the cooperative controller can be improved to make it closer model the a exoskeleton. An extra degree if freedom can be added to the system. Additional better motors and encoders should be used to improve the system accuracy and control. By using better sensors and motors the system can be controlled more precisely to allow better tracking and command signals. Additionally the system can be tested with different masses to test robustness.   

The cooperative controllers can be applied to robotic surgical systems such as the Master Tool Manipulators (MTM). Surgical movements can be learned and encoded for reproduction by having the expert teach the robot the trajectory of the surgical procedure and subsequently having the student be guided through the procedure by the robot. Additionally, the LfD can assist a robotic arm to help lift, carry and manipulate heavy objects. As stated above, the research and methods presented in this dissertation have a wide range of applications outside exoskeletons research.  




% An area of interest worth exploring is the kinematics and dynamics of the crutch and how they are used during a gait cycle; this could prove interesting for understanding the dynamic load on the person's body.




   