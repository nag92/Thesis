\section{Future Work}

The future work for this research includes completing the physical exoskeleton system. The Maxon motors need to be integrated with the harmonic gearbox and with LARRE's hips and knees, allowing the exoskeleton to be used actively. The Xilinx control board also needs to be completed and programmed; while this board is built, the motor drivers need to be integrated. Additionally, low-level code needs to be developed to control the motors. Additionally, an extra un-actuated degree of freedom should be added to the hip to allow some side-to-side motion, which would help directional change and turning. 

Once the motors and controllers have been integrated into LARRE, a study on normally abled people should be conducted. The cooperative controllers need to be tested and compared to the simulation models. This will also require writing the code for the controller to run on real-time OS on an embedded system. All human studies should be conducted with extreme caution to avoid injury. 

The bio-knee is a promising area of study. By studying MRI scans and motion capture trails of healthy patients, the knee can be customized to the individual. The MRI can show the difference in alignment between a pin joint knee and the customized bio-knee model. Additional research needs to be done to examine the relationship of femur and tibia head and shapes to height and gender so that expensive MRI do not necessary have to be conduct to get the correct scaling parameter. Additional is the limitation of mocap can be overcome, that could also be used as an effective measuring of measuring the rotation to translation relationship. However, this will be complex do the inherent nature of skin based marker. The manual palpation errors and skin movement noise will have to be overcome. Additionally there are problems when calculating the instance center of rotation with frame to frame marker locations.  

The simulation framework can be included to allow ground-level walking. The AMBF simulation framework is still new and under constant development, with new bugs being discovered and updates being pushed constantly. The new updates may be used to fix the frictional problem between the exoskeleton's feet and the ground plane. once this is complete more complex controller can be tested in the simulation environment. The controllers should be able to handle the ground force reactions since they are robust to disturbance. 


The cooperative controllers and LfD methods presented in this dissertation have board applications out the field of lower limb exoskeletons including robotic surgical system and industrual robotic systems. The cooperative controllers can be applied surgical robotic systems such as the the master tool manipulators (MTM). Surgical movements can be learned and encoded for reproduction. It can be used for training a user where it applies forces to their hand to follow a learned trajectory of an expert. Additionally, it can be used to build robotic arms to help lift and carry and manipulate heavy objects. The LfD can be used to teach complex trajectories for manipulation. Where the cooperative controller can be used to help a person lift the object. 




% An area of interest worth exploring is the kinematics and dynamics of the crutch and how they are used during a gait cycle; this could prove interesting for understanding the dynamic load on the person's body.




