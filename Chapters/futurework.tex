\section{Future Work}

The future work for this research includes completing the physical exoskeleton system. The Maxon motors need to be integrated with the harmonic gearbox and LARRE's hips and knees, allowing the exoskeleton to be used actively. The design for the motor plates in complete but needs to be manufactured. The motor coupler between Harmonic gearbox and Maxon motor has been designed. Additionally, an extra un-actuated degree of freedom should be added to the hip to allow some side-to-side motion, which would help directional change and turning. 


The Xilinx control board also needs to be completed and programmed; while this board is built, the motor drivers need to be integrated, and low-level code needs to be developed to control the motors. 

Once the motors and controllers have been integrated into LARRE, a study on typically normally able people should be conducted. The cooperative controllers need to be tested and compared to the simulation models; this will also require writing the code to run on real-time OS on an embedded system. All human studies should be conducted with extreme caution to avoid injury. 

The bio-knee is a promising area of study. By studying MRI scans and motion capture trails of healthy patients, the knee can be customized for each individual. The MRI can show the difference in alignment between a pin joint knee and the customized bio-knee model. Additional research needs to be done to examine the relationship of the femur and tibia head shapes to the height and the gender so that expensive MRIs do not necessarily have to be conducted to get the correct scaling parameter. Additionally, the limitation of mocap needs to be overcome to measure the rotation to translation relationship effectively. However, this will be complex to accomplish due to the inherent complications of skin-based markers and the difficulty of calculating the instantaneous center of rotation with the frame to frame marker locations.  

Additionally the knee should be dynamically tested. Using a motion capture system and the force plates. The dynamics and kinematics of the knee can be measured. An IRB has to be written for this study. Strap connectors have to be designed and built to attach to the person. The IRB trial should include approximately ten subjects walking across the room. By comparing the person walking with and without the bio-knee, the gait cycles can be compared. 

The simulation framework can be included to allow ground-level walking. The AMBF simulation framework is still new and under constant development, with new bugs being discovered and updates being pushed constantly. The new updates may be used to fix the frictional problem between the exoskeleton's feet and the ground plane. Once this is complete, more complex controllers can be tested in the simulation environment. The controllers should have the ability to handle the ground force reactions since they are robust to disturbance. 


The cooperative controllers and LfD methods presented in this dissertation have broader applications out of the field of lower limb exoskeletons, including robotic surgical systems and industrial robotic systems. A two Dof penulumn should be developed to simulate and test the controller on a system that closer resembles a leg. This system can be improved over the presented system by using better motors with more accurate encoders. This will allow finer control over the robotic system. 

The cooperative controllers can be applied to robotic surgical systems such as the Master Tool Manipulators (MTM). Surgical movements can be learned and encoded for reproduction by having the expert teach the robot the trajectory of the surgical procedure and subsequently having the student be guided through the procedure by the robot. Additionally, the LfD can assist a robotic arm to help lift, carry and manipulate heavy objects. As stated above, the research and methods presented in this dissertation have a wide range of applications outside exoskeletons research.  




% An area of interest worth exploring is the kinematics and dynamics of the crutch and how they are used during a gait cycle; this could prove interesting for understanding the dynamic load on the person's body.




   